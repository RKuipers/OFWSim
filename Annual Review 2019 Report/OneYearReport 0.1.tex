\documentclass[a4paper,12pt]{article}
\usepackage[english]{babel}
\usepackage{graphicx}
\graphicspath{ {images/} }
\usepackage[font=it,labelfont=bf]{caption}
\usepackage{algorithm}
\usepackage{algpseudocode}
\usepackage{varwidth}
\usepackage{amsmath}
\usepackage{amsthm}
\usepackage{subcaption}
\usepackage{float}
\usepackage{titlesec}
\usepackage{cleveref}
\usepackage{cite}
\usepackage{url}

\linespread{1.3}

\captionsetup[subfigure]{subrefformat=simple,labelformat=simple}
\renewcommand\thesubfigure{(\alph{subfigure})}

\setcounter{secnumdepth}{4}
\newcommand{\myparagraph}[1]{\paragraph*{#1}\mbox{}\\}

\newtheorem{theorem}{Theorem}[section]
\newtheorem{lemma}[theorem]{Lemma}
\newtheorem{defin}{Definition}
\newcommand{\mydef}[3]{
\begin{defin}
\textsc{#1}

Given: #2

Question: #3
\end{defin}}

\newcommand{\bigO}[1]{$\mathcal{O}$($#1$)}
\newcommand{\bigOs}[1]{$\mathcal{O^*}$($#1$)}
\newcommand{\NP}{$\mathcal{NP}$}
\newcommand{\acco}[1]{\{ #1 \}}

\newcommand{\vecarr}[3]{\overset{#2}{\overrightarrow{#1}}}

%Door deze regel springt de eerste regel van
%elke alinea niet meer steeds een stukje in.
%\setlength{\parskip}{\baselineskip}
%Door deze regel wordt tussen de alinea's steeds
%een regel overgeslagen.

%\setlength{\columnseprule}{1pt}
%\def\columnseprulecolor{\color{blue}}

\newcommand{\algorithmicbreak}{\textbf{break}}
\newcommand{\Break}{\State \algorithmicbreak}



\begin{document}
\title{Simulation and Optimisation of Offshore Renewable Energy Arrays for Minimal Life-Cycle Costs}
\author{Robin Kuipers \\[1cm] Supervisors: \\ Kerem Akartunali \\ Euan Barlow\\[2cm] University of Strathclyde \\ Strathclyde Business School \\ {\small Glasgow, Scotland}}
\date{April, 2019}

\maketitle

\pagebreak

\begin{abstract}
This report aims to give a detailed overview of the research progress I made between March 2018 and March 2019, the first year of this project. It will explain the research subject, give an overview of the relevant literature I've read and the limitations of the current research. Furthermore, it will discuss the work I've done on a simulator, to be used for later research. Finally, my future research on this topic will be estimated. 
\end{abstract}

\pagebreak

\tableofcontents

\pagebreak

\section{Introduction} \label{s:intro}
Over the past year, I have been researching scheduling related to offshore windfarm projects; the installation, maintenance, and dismantling of windfarms in various seas, primarily the North Sea. The installation and dismantling projects can often take up to several years, and the lifespan of such farms is about 30 years, over which maintenance has to be done. For these projects, expensive vessels have to be used, the rent of which is often upwards of \pounds 100.000 per day\cite{barlow2014support}, hence a project with multiple vessels over many months can cost upwards of \pounds 100 million \cite{kaiser2010offshore}. Therefore, even small improvements to the schedules can save significant amounts of money.

Because of that, I expected a lot of research in this area had already been conducted, but this turned out to be less the case than I would have expected. Most research is fairly recent, and there are significant literature gaps. The primary obstacle in this scheduling problem which separates it from more traditional scheduling problems is the high degree of non-deterministic factors, mainly related to the weather conditions. These projects take place on open sea, where weather can often be rougher than on land. In addition, the high-tech vessels are performing operations on big industrial constructions, so they have a limited range of allowed wind speeds and wave heights. Something else which can further limit possible schedules is the inflexibility involved in vessel rental, since vessels of the required caliber cannot be rented on short notice, and often need to be rented for at least some minimum amount of time \cite{kerkhove2017optimised}, so adaptive in-the-moment scheduling is impossible and we cannot simply wait for an expected period of good weather based on real-time data to rent the vessels; if we expected some period to have good weather but this turns out not to be the case, we'll often have the vessels rented but unusable. This means the problem has a lot of nondeterministic yet impactful factors which any good solution would have to take into account. 

In this report I will talk about my progress and what I have learned over the past year. First, in \Cref{s:project} I will explain the examined problem in more detail. After that, in \Cref{s:lit}, I will recount the reading I have done over so far, which was the main focus of my work over the past year. This will include both research on this specific project, and research in the more general fields of (stochastic) scheduling and common methods such as optimisation and simulation. This will be the main focus of the report, as it was the main focus of my work. In addition to reading up on the state of the current research, I have also started building a Simulator for this problem, which I will discuss in \Cref{s:sim}. Finally, I will summarize my current standing and plans for future work in \Cref{s:concl}. 

\pagebreak

\section{The project} \label{s:project}
In this section I will attempt to summarize the different views I've seen on the project, in enough detail for the reader to understand the rest of the report. I will primarily talk about installation projects, as most of the literature I've read is about those problems. It is likely the dismantling projects are very similar in structure; anything that was build needs to be taken apart. Maintenance projects however will be very different, as they are over a much larger time scale and while the two other types of projects have a set number of tasks that would ideally be completed as fast as possible, the maintenance has a set duration, with a potentially varying number of tasks.

A typical offshore windfarm (OWF) will have two types of structures: Wind Turbine Generators (WTGs) and Offshore Substation Platforms (OSPs). The WTGs are the actual turbines generating energy, and the OSPs are hubs where the power generated in the WTGs is gathered and transformed before being transported to shore. A typical OWF currently in development might have upwards of 100 WTGs and 2 OSPs \cite{ruk2017, barlow2018mixed}. Each of these structures has a set of tasks that need to be completed in series for each individual structure, like preparing the sea surface, laying the foundation, installing the structure and laying the cables (each of these tasks may be split into more tasks such as loading, transport and installation) \cite{kerkhove2017optimised}. Within the group of tasks for a structure, most tasks will have to be performed in a specific order, while the tasks for seperate structures can be completed in any order. This is essential for some of the more sophisticated objective functions used in the literature; if one OSP and a set of WTGs connected to it are active, they can start generating power while the rest of the wind farm is still under construction \cite{barlow2017using}. This would mean the farm starts making money significantly earlier, and can therefore have big impacts on desired schedules. In some situations the company might even set desired deadline for certain milestones, like 10\% or 50\% of the WTGs being operational. 

%TODO: For the cites, look at sources and where they got it from; citing directly leads me to more (varied) sources

The measure with which to compare schedules can vary. The most basic measure would be to produce a schedule with the earliest end time. Given the stochastic nature of the project, naively this would become the earliest expected end time. There are however a lot of different goals that could be described; instead of expected end time, a high confidence might be desired for the specific project. So instead of looking at the expected end time, one could for example look at the end time by which we can be 95\% certain the project would be completed (based on simulations). Using the end time as a measure is also up for debate, as the net profit is often more important for the companies doing these projects. So while the end time would obviously be postponed, completely halting the project over the winter months (during which no rent would be payed for the vessels) might be beneficial for the total costs, since there will be less individual days during summer where the weather would require the work to be halted (while the vessels would still be rented). As I will be discussing a lot of the literature in \Cref{s:lit} I will discuss the goals and objective functions in more detail there. 

\pagebreak

\section{Literature} \label{s:lit}
In this section I will give an overview of the literature I have read to-date for this research. I will discuss some actual attempts to produce a schedule for the installation problem in \Cref{sss:sched}, and some research on the maintenance of the site will be shown in \Cref{sss:maint}. Other research related to offshore projects is shown in \Cref{sss:offsh}. Finally, I will highlight some other studies to do with uncertain scheduling in \Cref{sss:stoch}.  Afterwards in \Cref{ss:meth} the common methods for these problems will be discussed separate from any particular research. 

\subsection{Research overview} \label{ss:rese}

\subsubsection{Studies on the full installation} \label{sss:sched}
For the research discussed here, I will discuss some models and methods used, and try and compare these studies. 

\bigskip

In \cite{barlow2018mixed} a mixed-method approach is used. They identify the strengths and weaknesses of both simulation and optimisation, and aim to use the methods together in a way that exploits the strength of each. First the simulation is used to determine some amount of delay before starting the project, as this can have sizable impact on the total duration of the project. Then, a mathematical programming model is used to construct a schedule from this starting point. The output of that optimisation model is then used in the simulation to give a view of the overall costs and uncertainty of the project. 

In the simulation model, a synthetic weather time-series model is used. Their model is described in full in \cite{dinwoodie2014operational} , and uses a correlated autoregression model which identifies underlying trends in the data over time and aims to predict future behavior over time. This seems to be a fitting and sophisticated model. However, in the optimisation model a more simplistic model is used. Tasks are given a minimal duration $d_{i, min}$ and a maximum increase $d_{i, inc}$ of that duration. The realized duration is given by $d_{i,  min} + z_i d_{i, inc}$ for some $0 \leq z_i \leq 1$. The programming model has two levels; the outer model maximizes $\sum_i z_i \leq \Gamma$ for some given limit $\Gamma$. The inner model then uses the fixed values $z_i$ to set the start times of each task in accordance with their deadlines and precedence relations. 

In effect, a $\Gamma$ is chosen to represent a certain percentage of tasks taking their maximum duration. For example, if there are 200 tasks and 10\% of them can take the maximum duration $\Gamma$ is set to 20, and this can either result in 20 tasks taking their maximum duration with all other tasks being completed in the minimum duration, or every task taking 10\% of its maximum extra time ($d_{i, inc}$). There are two key aspects of weather delay not taking into consideration here. The first is that tasks can often not be performed at all during certain weather conditions. Resources found and discussed in other research (which will be discussed later in this section as well) mention maximum wave heights and wind speeds, which when reached could shut down installation (or at least certain tasks) for hours or maybe even days. These would be identified later when the schedule is tested in simulation (through the synthetic weather series), but it seems valuable if this behavior was incorporated in the optimisation model as tasks not being able to be performed is fundamentally different behavior than tasks simply taking longer. The second aspect not used in this model is that delays often occur time-dependent of each other. The model uses multiple vessels; if weather conditions at a certain time cause one vessel to experience delays, chances are the other vessels are experiencing similar delays at that time. Additionally, if a certain task is performed with some delay, the next task that vessel works on probably happens in similar weather conditions and may experience delay again. Therefore it seems a more sophisticated model which incorporates these characteristics of delays could yield results which better correspond to the reality of the projects. 

An additional criticism to make this research more applicable is the objective of the optimisation; currently the makespan of the longest path of the schedule is minimized, but a company doing the installation will generally be more interested in the total cost of a schedule. This is clearly very related to the makespan, but can differ if certain vessels are used for longer than others, and when parts of the OWF can become active (and generate profit) earlier. Early profits are currently treated as certain milestones to reach, but when using the monetary value of the project in the objective function this can be done more naturally. The common practice of discounting future spending and profit could also be incorporated in this to more accurately represent the interests of the company. It seems like more research into this could therefore be beneficial. 

Finally the strengths of the simulation could be exploited more. In the research presented, a start date is determined at the start and this is then used in the optimisation and for the rest of the research. Optimal schedules starting in each month could be explored more, as well as pausing the operations after they started. For example, if the winter months turn out to be very difficult and slow to work in, the project could be paused over several months, over which no rent for the vessel would have to be payed. This would seem to be particularly effective if a subset of the WTGs could be operational before pausing, as they would generate energy (and revenue) during winter. These seems to be possible areas of research which could be valuable to explore in the future.  

\bigskip

A different model is designed in \cite{kerkhove2017optimised}. It uses similar methods as used in \cite{barlow2018mixed}, namely a mixed-method of optimisation and simulation, but but in a significantly different way. The research incorporates certain industry restrictions to do with vessel renting; it keeps track of when vessels are commissioned and decommissioned, incorporates a price for both, and has minimal renting and downtime periods. These logically make sense, as the vessels likely have costs to travel between the storage and used port, and the rental company will usually not be able to find other uses for a vessel if it has a sort down time. 

Related to this, the variables being decided on are the gate times of the tasks, instead of the start times as previously. The gate time refers to the time from which resources for a certain task should be made available. These times should be determined in advance, as there are generally long lead times associated with renting the vessels (hence they do not lend themselves much for dynamic scheduling). This allows for the optimisation to directly influence when the money is spend, which is a major focus of this research. The objective function is also different to accommodate this, it maximizes net present value of the complete project. This takes into account discount rates both on expenses and project value, which are emphasized in the literature. These rates are higher than traditionally used, as a main reason to include discount rates is the inherent uncertainty of the future, and projects of this nature have an even higher degree of uncertainty.

This leads me to a major limitation of this research. While the monetary value of the project is presented as the main focus, there is no mention of any positive cashflow starting when the project is partly done, nor are there any deadlines for milestones mentioned (such as when the first 10\% or 50\% of the WTGs are operational). Therefore the only positive value considered in the objective is the fixed value of the project upon completion, subject to discount rates. For research so focused on the financial perspective, not incorporating the possibility of early positive cashflows seems like a big gap, and a possible improvement for future research.

The actual methods used in the research are simulation and optimisation (primarily through local search). For the local search, two types of genes are used to represent restrictions of time periods in which certain activity types can be performed. For example, a gene could restrict all activities between December and February, and only allow for certain activities (less susceptible to weather) between September and November. The difference between the two gene types is that one type distinguishes between the years and the other does not (the former might have different weather restrictions for the first July and the second July of the project, while the latter sees them as the same). A simulated annealing algorithm is used to optimise the values of the genes, which are then converted into gate times (through a fairly straightforward algorithm) and tested in a simulation where the net present value of the project is calculated over several runs with different weather series. As an alternative to the local search, a heuristic is used where all activities are seen as the same type and a set of consecutive months to totally halt all execution are identified. Since there is only one month as the start of the `break' and one month as it's end, there are only $12^2 = 144$ combinations, which are all tested in simulation. In the case study for which experiments were run, this dedicated search outperformed both variations of local search. A way to possibly improve this optimisation is to first use the dedicated search to identify regions of the solution space which yield good results, and then use the best results of this search as starting solutions for the local search (where activity types are taken into consideration again). 

Another limitation of this paper are the genes, which are fairly high level. They determine which (category of) tasks can be performed in which month, but determining the schedule on a more detailed timescale can likely lead to much better results. A conversion algorithm is used to generate gate times for each task based on any given chromosome, but it is very simplistic. Every task is placed immediately after all preceding tasks should be completed if there are no weather delays. This is then tested with a single simulation, and if any task executions violate the given chromosome the gate time of that task is shifted forward to the next eligible period. This seems like a very simplistic method to use (as the genes operate on a much larger timescale than the gate times) and leads me to think improvement might be very beneficial.

The final thing to note about this research is the advanced weather simulation model used for the simulation. It uses a combination of transition probability matrices and a Weibull distribution to generate wind speeds and wave heights that are both sufficiently correlated to each other and themselves. The data this is based on was divided by month of the year, so the model takes time of year into account. Given the degree to which this type of project depends on the weather conditions, a realistic and sophisticated model like this is beneficial for the applicability of the model. 

FOR SOME OF THE PAPERS LISTED BELOW I HAVE THE NOTES ABOUT WHAT I WANT TO SAY BUT I HAVE NOT YET WRITTEN IT DOWN. THEY ALL WILL BE SHORTER THAN WHAT I DID FOR THE PREVIOUS PAPERS BUT I TRY TO GET TO THE HEART OF THE MODELS USED AND THE LITERATURE CONTRIBUTIONS AND GAPS

%TODO: consider swapping sections 3.1.2 and 3.1.2  if 3.1.3 is only about the installation (and possibly rename both/either)

%TODO
\subsubsection{Maintenance of the site} \label{sss:maint}
Lorem Ipsum \\ 
\cite{dinwoodie2012analysis} \\
\cite{merigaud2016condition} \\

%TODO
\subsubsection{Related research regarding offshore projects} \label{sss:offsh}
A group of researchers at my department, whose research \cite{barlow2018mixed} I have already described in \Cref{sss:sched}, have developed a simulation tool with which various logistical decisions for the installation of an OWF have been explored. In \cite{barlow2014support} the number of installation vessels and supply barges is experimented with, as well as their relative release dates. These simulations consider only the installation of OSPs and not WTGs. There are several measures considered for the quality of a setup such as the total cost of the project and the completion dates of the first and last OSPs, as well as the uncertainty of those measures (shown through the median and 90\% confidence interval for these measures, resulting from 100 simulations). \\
In \cite{barlow2014assessment} the same tool is used to compare several types of installation vessels. Simulations are run with vessels varying in four measures; number of WTGs it is able to carry at once, transition speed, maximum wave height for transitioning and maximum wave height during jacking operations. The values tested range from current vessels to expected vessels available in the near future. The transitioning speed is the only measure which seems to yield continuing improvement; the other three measures have clear diminishing returns and improvement beyond vessels currently available is unlikely to be worth it (given that newer vessels will have a higher cost). A criticism of this research is that all measures are tested in isolation, while three of the four measures are directly related the number or length of transits between the port and the OWF site. It is clear that the average transitioning speed matters less when the capacity is high enough to limit the number of transitions. However, the simulation tool used for this is capable of simulating those vessels, so if used to support decisions in an actual installation project the potential vessels used can be compared. \\
In \cite{barlow2017using}, the final paper (so far) on this tool, they go a bit more in depth on the tool and simulation as a method for supporting decisions in these projects. They talk about how the model was constructed and how after each round of improvements the model was validated with industry partners to ensure its accuracy. Some experiments are also discussed, the first being a simple simulation of the installation of a fictional OWF over 1000 weather series. The costs and duration of each segment of operations are analysed, and it is shown desirable to reduce any uncertainty as distribution of costs has a long tail. The standard deviation is \pounds 11.80M (approximately 5\% of the \pounds 233.89M median cost) so reducing uncertainty could lead to saving significant amounts of money. Then experiments with different number of supply barges are performed for the WTG jacket phase of the installation. A scenario is set up where these operations start a full year after the preceding phase of laying the foundations to ensure no delay is caused by that. Two sets of experiments are done, one with a single installation vessel and one with two installations vessels. For both scenarios similar patterns are found in the result; for the costs, there is an optimal minimum and increasing the number of supply barges from there slowly increases cost. The duration of the installation tends to only decrease very slowly from that optimal minimum as well. The statistical significance of their results are also tested, and some differences are found not to be statistically relevant. Finally, the impact of the relative starting dates of piling the foundations and starting jacket operations are explored. The start of laying the foundations is fixed at April 1st to optimally make use of the summer weather conditions. If the jacket operations would start at the same moment, there would be a lot of waiting and delays since the foundations for a specific WT need to be completed before the jacking operations can begin. This increases costs, so delaying the start of these latter operations is beneficial. A good amount of delay found is about 90 days, as waiting longer means more severe winter weather conditions. However, if it is possible to wait out the winter and delay operations about 420 days the total cost can be minimized as waiting this long means there are no delays due to waiting for the foundations, and making optimal use of summer weather conditions. The downside of this is a significant increase in makespan of the entire project. The OWF developer will have to decide between these options. For a different OWF the actual values will obviously differ, but similar patterns could be found. All in all, this tool is shown to be powerful and useful for logistical decisions related to the installation projects. 

\cite{leggate2010crew} \\
\cite{perez2013offshore}

%TODO
\subsubsection{More general work on stochastic scheduling} \label{sss:stoch}
Lorem Ipsum \\
\cite{herroelen2005project} \\
\cite{sevaux2002genetic} \\
\cite{artigues2000polynomial}

\subsection{Methodology} \label{ss:meth}
In the above literature, two main types of methods are used for this problem; optimisation and simulation. Often both are used together, either by tackling one part of the problem with simulation and another with optimisation (as in \cite{barlow2018mixed}), or by creating schedules with an optimisation model, which are then evaluated using simulation (as in \cite{kerkhove2017optimised}). In this sections the methods will be discussed in more detail, separate from the research they were used in. Other possible methods for problems of the kind we are looking at will also be discussed. 

\bigskip

\subsubsection{Simulation} \label{ss:sim}
Simulation, especially Discrete Event Simulation, is a good method to evaluate the strength of a proposed schedule under uncertain circumstances. If enough simulations are run, an accurate view of the realistic duration of a schedule can be gained. This does not only mean expected (average) duration, but also for example a confidence interval for the extreme cases. This does however have two big bottlenecks; the information gained through simulation can only be as precise as the weather model used for it, and doing enough simulations can take considerable time. 

\bigskip

If the weather data is limited in precision or amount of data available, the results of the simulation in turn will also be limited. First of all is the precision of the data; a simulation as detailed as we would like for this type of project would have timesteps in the range of 15 minutes to an hour. A lot of weather data has much larger timesteps, often of several hours. There are methods to interpolate between weather data points, but this tends to not be very precise as there are a lot of factors to take into account. The most important measures of the weather are wind speed and wave height, as by these measures the usability of the vessels can be determined. In order to have this data be realistic, the two measures have to be correlated, as well as have autocorrelation over time (the weather does not often drastically change in the span of 15 minutes). Having those values be realistically interpolated between two measured points is a difficult task, and something we will have to figure out a way to realistically handle for future research.

The amount of data available is also a common problem. A major reason for this is the natural change of weather depending on the time of year. Generally, a year would be split up in sections, for example the four seasons or the twelve months. Intuitively this makes a lot of sense; if we want to model the weather in January data from July is irrelevant for this. But the consequence is the sheer amount of time it takes to collect accurate data for each time period; if we want 300 days of data for the month of April, we need data collected over the span of 10 years. In addition, the conditions of the location the data is gathered from have to be similar to the location of the OWF. An important parameter for this is the distance to the shore, but there are more factors which play a part. 

A method to handle these drawbacks of the weather data is to not use raw data, but analyze the periodic characteristics of the change of both wind and wave states. Using these characteristics, one can generate weather data by drawing a random real data point within the desired time period, and extrapolating the data from there using the found characteristics. This is used frequently in the literature, and has been shown to have a good potential for stochastic simulations. 

Another method we are considering to explore relates more to robust optimisation. It might not be necessary to have a large number of simulations for all possible weather conditions. If, for each time period, we create a range of possible wind and wave states (including the extreme cases from the data) and run simulations for a set of (realistic) combinations of them we could aim to make a robust schedule that performs well within even the most extreme weather scenarios without having too many simulations to run. The drawback of this robust approach is a schedule that performs well in 99\% of weather cases might be discarded for the most extreme 1\% of cases. This can however be reduced by carefully selecting the data to use for the simulations, or allowing a certain number of bad scenarios (especially if the weather causing it is very rare). We might explore this more in future research. 

%TODO: Citations?

\bigskip

The other bottleneck, that of the considerable time it will take to run enough simulations, is something that will need to be taken into consideration for any research project using simulation. There are several methods to reduce the time used if necessary. A rigorous but effective step is is increasing the size of a timestep. This reduces accuracy, but does significantly reduce calculation time as well. Another is to carefully select input so less simulations need to be run for precision, as was discussed above. Aiming to get a robust result will generally need much less simulations than a result with simply the lowest average duration, as a lot of simulations are needed to calculate that average. 

There are many other possible methods to reduce number of simulations required, and which to apply depends on the specific research project and goals, and how much speed is needed. This is something we are aware of for when experiments will be done in the future. 

\bigskip

\subsubsection{Optimisation} \label{ss:opt}
Optimisation, in particular through linear programming, is an essential tool in problems like this, and is used extensively in the literature. Linear programming is preferred as non-linear programming is computationally much more complex to solve, therefore non-linear models are restricted in size in order to be solved in feasible time. However, both methods are used in the literature.

Even linear programming models computationally complex to solve, hence many methods are used to reduce to complexity of the models. There are standard techniques like Lagrangian relaxation \cite{fisher1981lagrangian} and column generation \cite{barnhart1998branch}, but there are more specific methods used as well. The primary way this is done is by limiting the scope of individual optimisation problems (not necessarily the research), by splitting up the research and using multi-level optimisation models, or using simulation for certain parts of it. 

Another method has been discussed before already; robust optimisation. As with Simulation, this reduces the number of considered inputs (primarily weather states, but also task duration which is sometimes stochastic) by only considering certain constructed uncertainty sets. This strongly reduces the number of situations considered in the optimisation. 

There are alternatives to computational programming. The most prominent is local search. A benefit of local search is that reasonably good results can often be found fairly quick, but this is connected to the main drawback of local search; it can often be difficult to get out of local optima. However, there is a lot of literature on local search and many variants designed to reduce its drawbacks. 

\bigskip

\subsubsection{Other Methods} \label{ss:otmet}
The main methods used in the literature are simulation and optimisation, but there are other relevant methods used in other scheduling problems. Evolutionary algorithms have been used for deterministic scheduling problems such as machine learning \cite{dorndorf1995evolution}. A good explanation of this technique for scheduling is given in \cite{cotta2007memetic}. While this technique seems to be fairly common in deterministic scheduling, not much research could be found in an uncertain setting. The exception is \cite{sevaux2002genetic} as discussed in \Cref{sss:stoch}. This research, although seemingly brief, gives quality results. Since there is no clear reason why evolutionary algorithms couldn't work well within an uncertain setting, this is a potential method to explore in the future. 

%Consider adding a Graph Theory section (unlikely as no real relevance)

\pagebreak

\section{Simulation tool} \label{s:sim}
In addition to getting caught up with the literature, I have also laid the foundations for some practical experiments. I have programmed the basics of an adaptable simulator from scratch. It is not yet operational and not very developed so far, but it can easily be expanded upon to fit future research. The use of this simulator would be checking feasibility of a generated schedule and analyzing the impacts of uncertain factors of the project. Currently it simulates a fixed set of different vessels that each have a schedule of tasks they aim to complete in the shortest timespan, but this can be changed based on the model design. In my current implementation, task duration is deterministic and independent of the weather conditions, but this will obviously be changed when an actual model is being tested. When a vessel cannot perform a certain task during the current weather conditions, it will wait at a designated spot, but it could also end up waiting near the location of its next task. The size of a single time step can be adjusted to most reasonable values without problems. 

%Paragraph about implementation (c++ and matlab)
The simulator has been implemented in C++, as it seems well suited for this. A benefit is that C++ can easily be combined with Matlab, which I am planning to implement the optimisation model in, as it is very suited for such uses and this will hopefully lead to a efficient implementation. While I am not yet proficient with Matlab, I have started learning it and I think I can then let the optimisation model in Matlab and the simulation model in C++ work together in a way useful for the model. 

%Paragraph about its uses
My goal for this simulator is for it to be usable as a tool in most of the likely research directions my PhD can go. The way it is currently set up the simulator is most suitable for fixed-task projects such as installation and decommission of the energy sites. However it is possible to adapt the simulator for other types of projects such as the maintenance of the site. In \Cref{s:concl} I will discuss various research directions; the simulator could be used for each of those directions. 

\pagebreak

\section{Future Work} \label{s:concl}
In this section I will outline two main directions this research could take and sum up the previously identified gaps in the literature and possible ways to fill them in. 

\bigskip

%Installation ideas: Improvement of Kerkhove with early starts and methodology improvements
The first direction is to focus on the installation projects and work from the existing literature. Of the models I discussed in \Cref{sss:sched} I identified both strengths and weaknesses, and my hope is to be able to combine them to highlight their strengths. I think a focus on cashflows is a good approach (as it will always be of interest to the project developer), but it should incorporate positive cashflows from partial completion, as well as possible milestone deadlines. Given this approach, it seems an investigation into both start date and off-months would be beneficial. A dedicated heuristic (as in \cite{kerkhove2017optimised}) could be used (where all activity types and years are treated equally) as base solutions, to then be potentially improved with a local search. However, this is not the only optimisation that should be performed. A restriction on which activities can be completed in which time periods is a starting point from which more detailed scheduling should occur. For each period a set of desired vessels should be determined, keeping in mind costs of (de)commissioning vessels and restrictions on timespans for that. A schedule should be made for each vessel such that each task is assigned a vessel and a timeslot. For this there are various optimisation techniques possible. The basic one to go to would be trying to solve a linear program, in which task durations can be the expected time within a time period (month) based on expected weather. A solution would then have to be checked through simulation. This may be difficult to do in a way that is computationally feasible in a reasonable time, but that is one of the challenges of this research, and there is a lot of literature on improving simulations when necessary. However, if this turns out to not be feasible there are alternatives such as using an evolutionary algorithm for optimisation. 

\bigskip

%TODO
%Maintenence ideas: New uncertain aspects to do with when stuff breaks (poisson distribution) possibly correlated with weather, ???
The second direction to take this research is to focus on the maintenance of an energy site, as discussed in \Cref{sss:maint}. WILL EXPAND ON THIS MUCH MORE AFTER WRITING 3.1.2 AS I DON''T HAVE FULL NOTES YET

\bigskip

%Combine for full life cycle 
There is a third direction following from the above two. In the event I spend some amount of time on both previous directions and gain reasonable results for both, I could aim to combine the models the techniques into a single model (and tool) for most scheduling decisions regarding projects of this kind. In addition to the above two directions, there is a third phase of the project to be taken into account for this; decommission. This is presumed to be very similar to the installation phase (hence not massively interesting in itself in my opinion) but there are some key differences, such as the reverse direction of cashflows (we start with the whole site operational but getting less operational over time), and very uncertain costs given that the decommissioning of sites currently being build will occur in several (unpredictable) decades. However, these obstacles can be overcome to make a good general use decision support tool for the scheduling of OWF projects. This is clearly a very ambitious goal, but it might be feasible depending on my research progress in the other directions over the following years. 

\bigskip

\subsection{Timeline and next steps} \label{ss:timel}
%Timeline and focus from here
Since research is inherently exploratory and unpredictable, it is impossible to have a timeline set in stone. However, in this section I will sketch a possible timeline and outline what I will focus on from here, based on the previously mentioned directions. 

My immediate focus will now move to the first direction discussed, that of the installation project. The plan is to focus on that for the next 6 months, and then review where to go from there. It is currently hard to say whether only following this direction will lead to a big and novel enough result for a PhD thesis. The individual components (uncertainty scheduling, aggregate scheduling, using net present value) have all been researched in the literature, but not in this combination. So the resulting model would be related to the models discussed in various papers, but their combination might be different enough to significantly transform the problem and its computability. The goal of the next 6 months is to do preliminary research in this direction and explore whether the new problem is far enough away from the existing literature. Depending on this result, a decision will be made whether to continue purely in this direction, or to shift to the third direction and include maintenance as well to create a general decision support tool, or to fully shift to the maintenance direction if the installation direction does not appear to be a valuable enough research area. 

A possible timeline for this would be:	

\begin{itemize}
	\item Year 2, first half: Finish basic simulator, design model for installation project, choose and implement optimisation algorithm for it
	\item Year 2, second half:  Contact industry companies for realistic project data, run initial experiments, design basic model for maintenance research, adapt simulator for it
	\item Year 3, first half: Run experiments with both installation and maintenance models, aim to combine them
	\item Year 3, second half: Test various alternative optimisation algorithms, finalize writing up research. 
\end{itemize}

This is the scenario in which I combine the installation and maintenance directions. Unmentioned in this timeline is possibly writing a paper on each individual model, and continues work on writing my thesis. In the case where I focus on either of the directions the timeline will look similar just more focused in either direction. 

\pagebreak

\bibliographystyle{alpha}
\bibliography{mybib}

%TODO: Uniformilize usage of I and We
%TODO: Uniformilize usage of optimisation/optimization and analyse/analyze

\end{document}