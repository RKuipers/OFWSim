\documentclass[a4paper,12pt]{article}
\usepackage[english]{babel}
\usepackage{graphicx}
\graphicspath{ {images/} }
\usepackage[font=it,labelfont=bf]{caption}
\usepackage{algorithm}
\usepackage{algpseudocode}
\usepackage{varwidth}
\usepackage{amsmath}
\usepackage{amsthm}
\usepackage{subcaption}
\usepackage{float}
\usepackage{titlesec}
\usepackage{cleveref}
\usepackage{cite}
\usepackage{url}

\linespread{1.3}

\captionsetup[subfigure]{subrefformat=simple,labelformat=simple}
\renewcommand\thesubfigure{(\alph{subfigure})}

\setcounter{secnumdepth}{4}
\newcommand{\myparagraph}[1]{\paragraph*{#1}\mbox{}\\}

\newtheorem{theorem}{Theorem}[section]
\newtheorem{lemma}[theorem]{Lemma}
\newtheorem{defin}{Definition}
\newcommand{\mydef}[3]{
\begin{defin}
\textsc{#1}

Given: #2

Question: #3
\end{defin}}

\newcommand{\bigO}[1]{$\mathcal{O}$($#1$)}
\newcommand{\bigOs}[1]{$\mathcal{O^*}$($#1$)}
\newcommand{\NP}{$\mathcal{NP}$}
\newcommand{\acco}[1]{\{ #1 \}}

\newcommand{\vecarr}[3]{\overset{#2}{\overrightarrow{#1}}}

%Door deze regel springt de eerste regel van
%elke alinea niet meer steeds een stukje in.
%\setlength{\parskip}{\baselineskip}
%Door deze regel wordt tussen de alinea's steeds
%een regel overgeslagen.

%\setlength{\columnseprule}{1pt}
%\def\columnseprulecolor{\color{blue}}

\newcommand{\algorithmicbreak}{\textbf{break}}
\newcommand{\Break}{\State \algorithmicbreak}



\begin{document}
\title{Simulation and Optimisation of Offshore Renewable Energy Arrays for Minimal Life-Cycle Costs}
\author{Robin Kuipers \\[1cm] Supervisors: \\ Kerem Akartunali \\ Euan Barlow\\[2cm] University of Strathclyde \\ Strathclyde Business School \\ {\small Glasgow, Scotland}}
\date{April, 2019}

\maketitle

\pagebreak

\begin{abstract}
This report aims to give a detailed overview of the research progress I made between March 2018 and March 2019, the first year of this project. It will explain the research subject, give an overview of the relevant literature I've read and the limitations of the current research. Furthermore, it will discuss the work I've done on a simulator, to be used for later research. Finally, my future research on this topic will be estimated. 
\end{abstract}

\pagebreak

\tableofcontents

\pagebreak

\section{Introduction} \label{s:intro}
Over the past year, I have been researching scheduling related to offshore windfarm projects; the installation, maintenance, and dismantling of windfarms in various seas, primarily the North Sea. The installation and dismantling projects can often take up to several years, and the lifespan of such farms is about 30 years, over which maintenance has to be done. For these projects, expensive vessels have to be used, the rent of which is often upwards of \pounds 100.000 per day\cite{barlow2014support}, hence a project with multiple vessels over many months can cost upwards of \pounds 100 million \cite{kaiser2010offshore}. Therefore, even small improvements to the schedules can save significant amounts of money.

Because of that, I expected a lot of research in this area had already been conducted, but this turned out to be less the case than I would have expected. Most research is fairly recent, and there are significant literature gaps. The primary obstracle in this scheduling problem which seperates it from more traditional scheduling problems is the high degree of non-deterministic factors, mainly related to the weather conditions. These projects take place on open sea, where weather can often be rougher than on land. In addition, the high-tech vessels are performing operations on big industrial constructions, so they have a limited range of allowed wind speeds and wave heights. Something else which can further limit possible schedules is the inflexibility involved in vessel rental, since vessels of the required caliber cannot be rented on short notice, and often need to be rented for at least some minimum amount of time \cite{kerkhove2017optimised}, so adaptive in-the-moment scheduling is impossible and we cannot simply wait for an expected period of good weather based on real-time data to rent the vessels; if we expected some period to have good weather but this turns out not to be the case, we'll often have the vessels rented but unusable. This means the problem has a lot of undeterministic yet impactful factors which any good solution would have to take into account. 

In this report I will talk about my progress and what I have learned over the past year. First, in \Cref{s:project} I will explain the examined problem in more detail. After that, in \Cref{s:lit}, I will recount the reading I have done over so far, which was the main focus of my work over the past year. This will include both research on this specific project, and research in the more general fields of (stochastic) scheduling, optimisation and simulation. This will be the main focus of the report, as it was the main focus of my work. In addition to reading up on the state of the current research, I have also started building a Simulator for this problem, which I will discuss in \Cref{s:sim}. Finally, I will summarize my current standing and plans for future work in \Cref{s:concl}. 

\pagebreak

\section{The project} \label{s:project}
In this section I will attempt to summarize the different views I've seen on the project, in enough detail for the reader to understand the rest of the report. I will primarily talk about installation projects, as most of the literature I've read is about those problems. It is likely the dismantling projects are very similar in structure; anything that was build needs to be taken apart. Maintenance projects however will be very different, as they are over a much larger time scale and while the two other types of projects have a set number of tasks that would ideally be completed as fast as possible, the mainenance has a set duration, with a potentially varying number of tasks.

A typical windfarm will have two types of structures: Wind Turbine Generators (WTGs) and Offshore Substation Platforms (OSPs). The WTGs are the actual turbines generating energy, and the OSPs are hubs where the power generated in the WTGs is gathered and transformed before being transported to shore. A typical windfarm currently in development might have upwards of 100 WTGs and 2 OSPs \cite{ruk2017, barlow2018mixed}. Each of these structures has a set of tasks that need to be completed in series for each individual structure, like preparing the sea surface, laying the foundation, installing the structure and laying the cables (each of these tasks may be split into more tasks such as loading, transport and installation) \cite{kerkhove2017optimised}. Within the group of tasks for a structure, most tasks will have to be performed in a specific order, while the tasks for seperate structures can be completed in any order. This is essential for some of the more sophisticated objective functions used in the literature; if one OSP and a set of WTGs connected to it are active, they can start generating power while the rest of the wind farm is still under construction \cite{barlow2017using}. This would mean the farm starts making money significantly earlier, and can therefore have big impacts on desired schedules. In some situations the company might even set desired deadline for certain milestones, like 10\% or 50\% of the turbines being operational. 

%TODO: For the cites, look at sources and where they got it from; citing directly leads me to more (varied) sources

The measure with which to compare schedules can vary. The most basic measure would be to produce a schedule with the earliest end time. Given the stochastic nature of the project, naively this would become the earliest expected end time. There are however a lot of different goals that could be described; instead of expected end time, a high confidence might be desired for the specific project. So instead of looking at the expected end time, one could for example look at the end time by which we can be 95\% certain the project would be completed (based on simulations). Using the end time as a measure is also up for debate, as the net profit is often more important for the companies doing these projects. So while the end time would obviously be postponed, completely halting the project over the winter months (during which no rent would be payed for the vessels) might be beneficial for the total costs, since there will be less individual days during summer where the weather would require the work to be halted (while the vessels would still be rented). As I will be discussing a lot of the literature in \Cref{s:lit} I will discuss the goals and objective functions in more detail there. 

\pagebreak

\section{Methodology} \label{s:meth}
In the literature, two main types of methods are used for this problem; optimisation and simulation. Often both are used together, either by tackling one part of the problem with simulation and another with optimisation (as in \cite{barlow2018mixed}), or by creating schedules with an optimisation model, which are then evaluated using simulation (as in \cite{kerkhove2017optimised}). Exactly how certain previous research uses these methods will be discussed more in \Cref{s:lit}, but a more general look at these methods, as well as other methods currently under consideration for this research, will be given in this section. 

\bigskip

\subsection{Simulation} \label{ss:sim}
Simulation, especially Discrete Event Simulation, is a good method to evaluate the strength of a proposed schedule under uncertain circumstances. If enough simulations are run, an accurate view of the realistic duration of a schedule can be gained. This does not only mean expected (average) duration, but also for example a confidence interval for the extreme cases. This does however have two big bottlenecks; the information gained through simulation can only be as precise as the weather model used for it, and doing enough simulations can take considerable time. 

\bigskip

If the weather data is limited in precision or amount of data available, the results of the simulation in turn will also be limited. First of all is the precision of the data; a simulation as detailed as we would like for this type of project would have timesteps in the range of 15 minutes to an hour. A lot of weather data has much larger timesteps, often of several hours. There are methods to interpolate between weather data points, but this tends to not be very precise as there are a lot of factors to take into account. The most important measures of the weather are wind speed and wave height, as by these measures the usability of the vessels can be determined. In order to have this data be realistic, the two measures have to be correlated, as well as have autocorrelation over time (the weather does not often drastically change in the span of 15 minutes). Having those values be realistically interpolated between two measured points is a difficult task, and something we will have to figure out a way to realistically handle for future research.

The amount of data available is also a common problem. A major reason for this is the natural change of weather depending on the time of year. Generally, a year would be split up in sections, for example the four seasons or the twelve months. Intuitively this makes a lot of sense; if we want to model the weather in January data from July is irrelevant for this. But the consequence of this is the shear time it takes to collect accurate data for each time period; if we want 300 days of data for the month of April, we need data collected over the span of 10 years. In addition, the conditions of the location the data is gathered from have to be similar to the location of the windfarm. The most important parameter for this is the distance to the shore, but there are more factors which play a part. 

A method to handle these drawbacks of the weather data is to not use raw data, but analyze the periodic characteristics of the change of both wind and wave states. Using these characteristics, one can generate weather data by drawing a random real data point within the desired time period, and extrapolating the data from there using the found characteristics. This is something we are planning to explore in future research, and we think it has potential to give a good weather model to be used for stochastic simulations. 

Another method we are considering to explore relates more to robust optimisation. It might not be necessary to have a large number of simulations for all possible weather conditions. If, for each time period, we create a range of possible wind and wave states (including the extreme cases from the data) and run simulations for a set of (realistic) combinations of them we could aim to make a robust schedule that performs well within even the most extreme weather scenarios without having too many simulations to run. The drawback of this robust approach is a schedule that performs well in 99\% of weather cases might be discarded for the most extreme 1\% of cases. This can however be reduced by carefully selecting the data to use for the simulations, or allowing a certain number of bad scenarios (especially if the weather causing it is very rare). We might explore this more in future research. 

%TODO: Citations?

\bigskip

The other bottleneck, that of the considerable time it will take to run enough simulations, is something that will need to be taken into consideration for any research project using simulation. There are several methods to reduce the time used if necessary. A rigorous but effective step is is increasing the size of a timestep. This reduces accuracy, but does significantly reduce calculation time as well. Another is to carefully select input so less simulations need to be run for precision, as was discussed above. Aiming to get a robust result will generally need less simulations than a result with simply the lowest average duration, as a lot of simulations are needed to calculate that average. 

There are many other possible methods to reduce number of simulations required, and which to apply depends on the specific research project and goals, and how much speed is needed. This is something we are aware of for when experiments will be done in the future. 

\bigskip

\subsection{Optimisation} \label{ss:opt}
Optimisation, in particular through linear programming, is an essential tool in problems like this, and is used extensively in the literature. Linear programming is preferred as non-linear programming is computationally much more complex to solve, therefore non-linear models are restricted in size in order to be solved in feasible time. However, both methods are used in the literature.

Even linear programming models computationally complex to solve, hence many methods are used to reduce to complexity of the models. There are standard techniques like Lagrangian relaxation \cite{fisher1981lagrangian} and column generation \cite{barnhart1998branch}, but there are more specific methods used as well. The primary way this is done is by limiting the scope of individual optimisation problems (not necesarilly the research), by splitting up the research and using multi-level optimisation models, or using simulation for certain parts of it. I will go into more detail on specific models in \Cref{s:lit}, where I will also give an example model from the literature to illustrate a way to model this type of project.

%TODO: Highlight example model

Another method has been discussed before already; robust optimisation. As with Simulation, this reduces the number of considered inputs (primarily weather states, but also task duration which is sometimes stochastic) by only considering certain constructed uncertainty sets. This strongly reduces the number of situations considered in the optimisation. 

\bigskip

\subsection{Other Methods} \label{ss:otmet}
While the main methods are Simulation and optimisation, there are more notable methods used in the litarature. The main one that stroke me as relevant are genetic algorithms. In \cite{kerkhove2017optimised} a local search algorithm is used on what they call `genes' (set of times from which a specific task is available) to find smal beneficiall changes to an existing schedule. This method seemed promising, and not a lot of research has been done in this area, hence it is being considered for use in our future research. Additionally, using the schedule performance as a measure of gene strength, potentially an evolutionary algorithm could be tested by combining successfull genes. As far as we know, this has not been used for research on this problem before and is therefore of potential interest in the future. This is strenghted by evolutionairy methods occasionally being found effective for a machine scheduling problem \cite{dorndorf1995evolution}. The windfarm installation problem could be modelled similar to a machine scheduling problem (where each vessel is a machine), hence this method may be beneficial.

%Consider adding a Graph Theory section (unlikely as no real relevance)

\pagebreak

\section{Literature} \label{s:lit}
In this section I will give an overview of the literature I have read to-date for this research. I will discuss some actual attempts to produce a schedule for the installation problem in \Cref{ss:sched}, and some research on the maintenance of the site will be shown in \Cref{ss:maint}. Other research related to offshore projects is shown in \Cref{ss:offsh}. Finally, I will highlight some other studies to do with uncertain scheduling in \Cref{ss:stoch}.  

\subsection{Schedulling the installation} \label{ss:sched}
For the research discussed here, I will discuss some models and methods used, and try and compare these studies. 

\bigskip

In \cite{barlow2018mixed} a mixed-method approach is used. They identify the strengths and weaknesses of both simulation and optimisation, and aim to use the methods together in a way that exploits the strength of each. First the simulation is used to determine some amount of delay before starting the project, as this can have sizable impact on the total duration of the project. Then, a mathematical programming model is used to construct a schedule from this starting point. The output of that optimisation model is then used in the simulation to give a view of the overall costs and uncertainty of the project. 

In the simulation model, a synthetic weather time-series model is used. Their model is described in full in \cite{dinwoodie2014operational} , and uses a correlated autoregression model which identifies underlying trends in the data over time and aims to predict future behavior over time. This seems to be a fitting and sophisticated model. However, in the optimisation model a more simplistic model is used. Tasks are given a minimal duration $d_{i, min}$ and a maximum increase $d_{i, inc}$ of that duration. The realized duration is given by $d_{i,  min} + z_i d_{i, inc}$ for some $0 \leq z_i \leq 1$. The programming model has two levels; the outer model maximizes $\sum_i z_i \leq \Gamma$ for some given limit $\Gamma$. The inner model then uses the fixed values $z_i$ to set the start times of each task in accordance with their deadlines and precedence relations. 

In effect, a $\Gamma$ is chosen to represent a certain percentage of tasks taking their maximum duration. For example, if there are 200 tasks and 10\% of them can take the maximum duration $\Gamma$ is set to 20, and this can either result in 20 tasks taking their maximum duration with all other tasks being completed in the minimum duration, or every task taking 10\% of its maximum extra time ($d_{i, inc}$). There are two key aspects of weather delay not taking into consideration here. The first is that tasks can often not be performed at all during certain weather conditions. Resources found and discussed in other research (which will be discussed later in this section as well) mention maximum wave heights and wind speeds, which when reached could shut down installation (or at least certain tasks) for hours or maybe even days. These would be identified later when the schedule is tested in simulation (through the synthetic weather series), but it seems valuable if this behavior was incorporated in the optimisation model as tasks not being able to be performed is fundamentally different behavior than tasks simply taking longer. The second aspect not used in this model is that delays often occur time-dependent of each other. The model uses multiple vessels; if weather conditions at a certain time cause one vessel to experience delays, chances are the other vessels are experiencing similar delays at that time. Additionally, if a certain task is performed with some delay, the next task that vessel works on probably happens in similar weather conditions and may experience delay again. Therefore it seems a more sophisticated model which incorporates these characteristics of delays could yield results which better correspond to the reality of the projects. 

An additional criticism to make this research more applicable is the objective of the optimisation; currently the makespan of the longest path of the schedule is minimized, but a company doing the installation will generally be more interested in the total cost of a schedule. This is clearly very related to the makespan, but can differ if certain vessels are used for longer than others, and when parts of the windfarm can become active (and generate profit) earlier. Early profits are currently treated as certain milestones to reach, but when using the monetary value of the project in the objective function this can be done more naturally. The common practice of discounting future spending and profit could also be incorporated in this to more accurately represent the interests of the company. It seems like more research into this could therefore be beneficial. 

Finally the strengths of the simulation could be exploited more. In the research presented, a start date is determined at the start and this is then used in the optimisation and for the rest of the research. Optimal schedules starting in each month could be explored more, as well as pausing the operations after they started. For example, if the winter months turn out to be very difficult and slow to work in, the project could be paused over several months, over which no rent for the vessel would have to be payed. This would seem to be particularly effective if a subset of the turbines could be operational before pausing, as they would generate energy (and revenue) during winter. These seems to be possible areas of research which could be valuable to explore in the future.  

\bigskip

A different model is designed in \cite{kerkhove2017optimised}. It uses similar methods as used in \cite{barlow2018mixed}, namely a mixed-method of optimisation and simulation, but but in a significantly different way. The research incorporates certain industry restrictions to do with vessel renting; it keeps track of when vessels are commissioned and decommissioned, incorporates a price for both, and has minimal renting and downtime periods. These logically make sense, as the vessels likely have costs to travel between the storage and used port, and the rental company will usually not be able to find other uses for a vessel if it has a sort down time. 

Related to this, the variables being decided on are the gate times of the tasks, instead of the start times as previously. The gate time refers to the time from which resources for a certain task should be made available. These times should be determined in advance, as there are generally long lead times associated with renting the vessels (hence they do not lend themselves much for dynamic scheduling). This allows for the optimisation to directly influence when the money is spend, which is a major focus of this research. The objective function is also different to accomodate this, it maximises net present value of the complete project. This takes into account discount rates both on exenses and project value, which are emphesised in the literature. These rates are higher than traditionally used, as a main reason to include discount rates is the inherent uncertainty of the future, and projects of this nature have an even higher degree of uncertainty.

This leads me to a major limitation of this research. While the monetary value of the project is presented as the main focus, there is no mention of any postive cashflow starting when the project is partly done, nor are there any deadlines for milestones mentioned (such as when the first 10\% or 50\% of the turbines are operational). Therefore the only postive value considered in the objective is the fixed value of the project upon completion, subject to discount rates. For research so focussed on the financial perspecitve, not incorporating the possibility of early positive cashflows seems like a big gap, and a possible improvement for future research.

The actual methods used in the research are simulation and optimisation (primarily through local search). For the local search, two types of genes are used to represent restrictions of time periods in which certain activity types can be performed. For example, a gene could restrict all activities between December and February, and only allow for certain activities (less susceptible to weather) between September and November. The difference between the two gene types is that one type distinguishes between the years and the other does not (the former might have different weather restrictions for the first July and the second July of the project, while the latter sees them as the same). A simulated annealing algorithm is used to optimise the values of the genes, which are then converted into gate times (through a fairly straightforward algorithm) and tested in a simulation where the net present value of the project is calculated over several runs with different weather series. As an alternative to the local search, a heuristic is used where all activities are seen as the same type and months to totally halt all execution are identified. Since there is only a binary value to be determined for each month there are only $12^2 = 144$ possible variations, which are all tested in simulation. In the case study for which experiments were run, this dedicated search outperformed both variations of local search. 

%Talk about limitations of conversion algorithm (between MAT/SAT and Gate genes)

%Talk about weather model

\bigskip

FOR SOME OF THE PAPERS LISTED BELOW I HAVE THE NOTES ABOUT WHAT I WANT TO SAY BUT I HAVE NOT YET WRITTEN IT DOWN. THEY WILL BE SHORTER THAN WHAT I DID FOR THE PREVIOUS PAPER BUT I TRY TO GET TO THE HEART OF THE MODELS USED AND THE LITERATURE CONTRIBUTIONS AND GAPS

\cite{barlow2017using}

\subsection{Maintenance of the site} \label{ss:maint}
Lorem Ipsum \\ 
\cite{dinwoodie2012analysis} \\
\cite{merigaud2016condition} \\

\subsection{Related research regarding offshore projects} \label{ss:offsh}
Lorem Ipsum \\
\cite{barlow2014assessment} \\
\cite{barlow2014support} \\
\cite{leggate2010crew} \\
\cite{perez2013offshore}

\subsection{More general work on stocahstic scheduling} \label{ss:stoch}
Lorem Ipsum \\
\cite{herroelen2005project} \\
\cite{sevaux2002genetic} \\
\cite{artigues2000polynomial}

\pagebreak

\section{Simulation tool} \label{s:sim}
Lorem Ipsum

\pagebreak

\section{Future Work} \label{s:concl}
THIS SECTION WILL MOSTLY SUM UP LITERATURE GAPS IDENTIFIED IN SECTION 4, SAY WHICH ONES IM MOST INTERESTED IN PURSUING (AND HOW) AND A POSSIBLE TIMELINE FOR THAT
Lorem Ipsum

\pagebreak

\bibliographystyle{alpha}
\bibliography{mybib}

\end{document}