\documentclass[a4paper,12pt]{article}
\usepackage[english]{babel}
\usepackage{graphicx}
\graphicspath{ {images/} }
\usepackage[font=it,labelfont=bf]{caption}
\usepackage{algorithm}
\usepackage{algpseudocode}
\usepackage{varwidth}
\usepackage{amsmath}
\usepackage{amsthm}
\usepackage{subcaption}
\usepackage{float}
\usepackage{titlesec}
\usepackage{cleveref}
\usepackage{cite}
\usepackage{url}
\usepackage{harvard}
\usepackage{thm-restate}
\usepackage[space]{grffile}

\citationmode{abbr}


\linespread{1.3}

\captionsetup[subfigure]{subrefformat=simple,labelformat=simple}
\renewcommand\thesubfigure{(\alph{subfigure})}

\setcounter{secnumdepth}{4}
\newcommand{\myparagraph}[1]{\paragraph*{#1}\mbox{}\\}

\newtheorem{theorem}{Theorem}[section]
\newtheorem{lemma}[theorem]{Lemma}
\newtheorem{defin}{Definition}
\newtheorem*{rquestion}{Question}
\newtheorem{subquestion}{Sub-Question}
\newcommand{\mydef}[3]{
\begin{defin}
\textsc{#1}

Given: #2

Question: #3
\end{defin}}

\newcommand{\bigO}[1]{$\mathcal{O}$($#1$)}
\newcommand{\bigOs}[1]{$\mathcal{O^*}$($#1$)}
\newcommand{\NP}{$\mathcal{NP}$}
\newcommand{\acco}[1]{\{ #1 \}}

\newcommand{\vecarr}[3]{\overset{#2}{\overrightarrow{#1}}}

%Door deze regel springt de eerste regel van
%elke alinea niet meer steeds een stukje in.
%\setlength{\parskip}{\baselineskip}
%Door deze regel wordt tussen de alinea's steeds
%een regel overgeslagen.

%\setlength{\columnseprule}{1pt}
%\def\columnseprulecolor{\color{blue}}

\newcommand{\algorithmicbreak}{\textbf{break}}
\newcommand{\Break}{\State \algorithmicbreak}



\begin{document}
\title{Simulation and Optimization of Offshore Renewable Energy Arrays for Minimal Life-Cycle Costs \\
\large Second Year Report}
\author{Robin Kuipers \\[1cm] Supervisors: \\ Kerem Akartunali \\ Euan Barlow\\[2cm] University of Strathclyde \\ Strathclyde Business School \\ {\small Glasgow, Scotland}}
\date{October, 2020}

\maketitle

\pagebreak

\begin{abstract}
This report aims to give a detailed overview of the research progress I made between January 2020 and September 2020, which is the time between the completion of the first yearly review and the start of the second yearly review of this PhD project into logistical decisions regarding offshore wind farms. It will quickly recap the progress made in the first year, and then detail the progress made in the second year, which primarily focused on developing and implementing models for optimization problems related to topic. The report also discusses future work, and what the next steps are. 
\end{abstract}

\pagebreak

\tableofcontents

\pagebreak

\section{Introduction}\label{ss:omtrp}
For the past two years, I have been researching logistics related to offshore windfarm projects; the installation, maintenance, and decommissioning of windfarms in various seas and oceans, primarily the North Sea. The installation and decommissioning projects (at the start and end of the windfarm's lifespan respectively) can often take up to several years, and the lifespan of such farms is usually between 20 and 30 years, over which maintenance has to be done. For these projects, expensive vessels have to be used, the rent of which is often upwards of \pounds 100.000 per day \cite{barlow2014support}, hence a project with multiple vessels over many months can cost upwards of \pounds 100 million \cite{kaiser2010offshore}. Therefore, even small improvements to the schedules can save significant amounts of money.

The complexity with these logistics comes from various factors, the first being the severe impact that weather conditions can have on when operational tasks can be completed. These projects take place on open sea, where weather can often be rougher than on land. In addition, the high-tech vessels are performing operations on large industrial constructions, hence there is a limited range of allowed wind speeds and wave heights. Another factor which can further limit possible schedules is the inflexibility involved in vessel chartering (renting), since vessels of the required caliber cannot be chartered on short notice, making adaptive in-the-moment scheduling is impossible. This means we will have to make decisions significantly in advance, and run the risk of chartering vessels in periods where they might not be able to complete their tasks due to the weather conditions. 

This report will detail the research progress I have made since the completion of my first yearly review, which (due to scheduling issues and delays) was only completed in January 2020. Further on in this section, I will recap the specific parts of this problem that I am looking at (\Cref{ss:prob}), and which research questions I initially posed (\Cref{ss:quest}). Then \Cref{s:model} through  \Cref{s:addit} will talk about the work I have been since the last review. Respectively, these sections are about the models I have developed, how I have implemented them, the knowledge gaps I closed, and all additional activities I have taken part in. Finally, \Cref{s:next} will discuss the next steps for this project. 

\subsection{Problem description} \label{ss:prob}
This research looks at offshore windfarms (OWFs) and the scheduling decisions made over the entire life-cycle of such an OWF. These windfarms generally consist of between 80-150 Wind Turbine Generators (WTGs) and the farms we focus on are located in the North Sea, roughly 50+ kilometers off the coast. A typical life-cycle would consist of roughly 2-3 yeas of installation, 20-30 years in which the WTGs generate energy and need to be maintained, and then another 2-3 years in which the turbines are decommissioned. These turbines are large structures that will need to be transported and installed by specialized, expensive vessels. 

\bigskip

In the literature these three phases are commonly treated separately. Both the installation and decommission phases have similar structures; a fixed set of tasks that needs to be completed as cheap or as fast as possible. Focusing on cost over speed might lead to a different schedule, as some decisions might slow down the overall project but lead to individual turbines being operational earlier, from which point it can start generating energy and income. Commonly there will be contractual or legal deadlines on installation and decommission, setting dates at which the OWF should be fully or partially operational, or completely decommissioned. A key difference between the installation and decommission phase are that in the installation phase minimizing cost will help reach the deadlines; ideally the installation is completed as soon as possible, as income will be generated by each completed turbine. For decommission this is not the case; if decommission starts later turbines will generate income for longer, but it will be harder to reach the deadlines. 

The maintenance phase has an entirely different structure from the other two phases. There will be a small set of fixed tasks, since there are legal requirements on minimum amount of maintenance that needs to be done, but this will not be the bulk of the work performed during this phase. A maintenance strategy will need to be formed that determines when turbines will be visited. These visits can be at predetermined moments, or after a turbine fails, or sometimes even right before a turbine is predicted to fail (based on sensor data). An optimal strategy will often use a combination of these types of visits in order to minimize cost, accounting for both the cost of repairs and the amount of income missed due to failures and downtime during maintenance. An additional difference during the maintenance phase is that tasks will often be smaller than the tasks in the other phases. During installation and decommission entire turbines need to be transported and worked on, often requiring very specific and expensive vessels. Conversely during maintenance many smaller repairs can often be done by simply sending engineers over and all you need is a crew transport vessel. This is not the case for every maintenance task as larger repairs and replacements will sometimes need to happen, but it will still hold for many small tasks.

\bigskip

Each of these three phases is subject to stochastic elements, the weather conditions being a major factor to take into account. In addition to the weather possibly restricting which tasks can be performed at any given time, it plays another factor during the maintenance phase. Strong winds also increase the energy output of the turbines, making it extra beneficial to have all turbines up and running before periods of strong winds. Other than the weather, task durations have an inherently uncertain factor, and turbine failures are also difficult to predict. All this together means that uncertain factors can have an immense impact on the overall costs of the project. For this reason robustness is often a metric to be taken into account as well. It might be in the operators best interest to work less efficiently if it reduces the chances of large delays.

\subsection{Research questions}\label{ss:quest}
In this PhD project my goal is to look at scheduling during the entire life-cycle of an OWF. As far as we are aware, this has not been done in the literature, which indicated there may be optimizations and new insights to be found here. Therefore my primary research question is:

\begin{restatable}{rquestion}{rquest}
Can considering the entirety of the life-cycle of an Offshore Wind Farm, and how each of the phases interact, improve logistical decision making on these projects?
\label{rquest}
\end{restatable}

It is clear that a primary reason for splitting the project into phases is that the problem becomes more manageable, and there is generally no need to make all scheduling decisions at the start of the project. There is no point in scheduling the entire decommission phase at the time of installation, as over the 20-30 year lifespan of the windfarm the available vessels will likely change. However, treating the phases entirely separately misses the interactions between the phases. This interaction exists within the real world, and if the literature ignores it this creates a divide between academia and the real world. For that reason I want to investigate these interactions.

\bigskip

The first type of interaction takes place when two phases are active at the same time; after the first turbines have completed installation they potentially need maintenance, while installation continues on the rest of the turbines. The reverse effect takes place when decommission starts, as it does not start at the same time for every turbine. During this time, the phases share resources and could potentially hinder each other (when the same port is used for different vessels servicing different phases). On the other hand, if attention is payed this sharing of resources could be beneficial. Vessels for installation can potentially serve as crew transport vessels in addition to their usual tasks, or a vessel can be used to do some maintenance and some decommission tasks. Therefore paying attention to this overlap period could both help reduce obstacles and create new benefits from this interaction. 

The second type of interaction is the long-term effect scheduling decisions might have. If the installation is looked at in isolation a schedule might be produced in which the completion time of the first and the last turbine is years apart; this might have an affect on their wear and chance to fail over the course of the maintenance phase. This in turn might also lead to those first-installed turbines being decommissioned first as well. Since decisions made during the installation phase might still influence events long after installation is complete, these long term effects might also influence the decision in the first place. For that reason this interaction can be looked at from two perspectives; early decisions that are influenced by their long-term effects, and later decisions that are influenced by decisions made in earlier phases. 

These interactions bring me to the sub-questions of my research:

\begin{restatable}{subquestion}{sqa}
\label{sqa}
Can considering how phases in the life-cycle of a windfarm overlap and share resources improve logistical decision making on these projects?
\end{restatable}

\begin{restatable}{subquestion}{sqb}
\label{sqb}
Can simulating the entire life-cycle of a windfarm provide useful data to base logistical decisions on in the later phases of these projects?
\end{restatable}

\begin{restatable}{subquestion}{sqc}
\label{sqc}
Can considering the long-term effects of logistical decisions early on in the life-cycle of a windfarm improve these decisions? 
\end{restatable}

\pagebreak

\section{Developing models}\label{s:model}
Lorem Ipsum 

\subsection{Initial models}
Lorem Ipsum 

\subsection{Combined models}
Lorem Ipsum 

\pagebreak

\section{Implementation}\label{s:impl}
Lorem Ipsum 

\pagebreak

\section{Closing knowledge gaps}\label{s:know}
Lorem Ipsum 

\pagebreak

\section{Additional activities} \label{s:addit} %COWORK course and teaching
In addition to the work directly related to my PhD project, I have participated in some courses. 

I was supposed to take part in the NATCOR course Convex Optimization, to be held at the University of Edinburgh in June 2020. However, due to the Covid-19 pandemic this course was canceled. I was also enrolled in the NATCOR course Forecasting and Predictive Analytics, which was to be held in September 2020 at Lancaster University. This course, due the same pandemic, has been postponed to February 2020. 

\bigskip

While those courses not taking place as planned was disappointing, another course I did not originally plan to go to had to move entirely online, making it possible for me to follow it. The course in question was CO@Work (Combinatorial Optimization at work), hosted by the University of Berlin. This two-week course offered lectures (via Youtube) on varying topics, and exercise and Q\&A sessions (via Zoom). Since the Zoom sessions took place at inconvenient times (due to timezones), I primarily partook in the lectures. The course was aimed at a wide variety of students, ranging from undergraduates new to optimization, to PhD students like myself. This meant that some lectures went over material I am already familiar with, like the workings of the simplex algorithm. Other material focused on techniques to help with solving linear and mixed-integer programs, such as column generation and branch-and-bound techniques. While I had previously been taught how these methods work, this was years ago, and the refresher was quite helpful. The course also had more time to go into details on these techniques, so I certainly learned new aspects of these techniques. Finally there were some corporate talks, at which various companies within the optimization industry talked about what they do and what a career with them could look like. While some of these talks seemed very specific to the company hosting it and less interesting, some also simply talked about work and careers as optimizers in general, which I found very helpful and interesting. 

Generally I am glad I got to follow this course, and the format of having a few hours of lectures to watch at my own pace helped lower the workload (compared to traveling to Berlin for two weeks). This allowed me to still work on my own project during this course, and since the lectures were recorded videos rather than live, I could rewatch the parts that were most interesting or most complex. That said, I did miss the social aspect of this course, as normally with courses such as this you get to spend a week or two with students from all over the world who all study subjects similar to my own. This dimension was entirely missing, which is of course a strong drawback. 

\bigskip

Apart from following courses, I have also helped teach a course. The course, Information Access \& Mining (CS412), was a 4th year Computer Science course focusing on data analysis through machine learning. This is fairly far removed from the topic of my own project, but I was still fairly able to teach the course because of my Computer Science background. The main thing that was new for me was the Python language used in the course, with which I was previously unfamiliar. However, this simply meant that in addition to the teaching experience I got from teaching this course, I also made myself acquainted with Python, which turned out to be a relatively easy language to learn. I lead the labs, which meant I had to answer students questions regarding their exercises. Since I prepared the labs well, answering these questions was fairly straightforward. Additionally I had to mark the exercises, which took the majority of my time spend. But since I was provided with an decently detailed answer key, this was not very difficult either. After the labs stopped (due to the Covid-19 pandemic) my work solely consisted of the marking, lowering the workload. 

This was my first real teaching experience, and I think it went very well. I enjoyed helping the students, and I enjoyed expanding my own knowledge of both the programming language and the subject matter. If I get another chance to help out with a course that interests me during my PhD, I will likely take it. 

\pagebreak

\section{Next steps}\label{s:next}
Lorem Ipsum 

\subsection{Timeline}
Lorem Ipsum 

\pagebreak

\bibliographystyle{agsm}
\bibliography{mybib}

\end{document}